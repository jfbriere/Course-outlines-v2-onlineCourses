\entry{Course content}{Movies and television shows sometimes take great liberties when it comes to the laws of physics or in portraying science, but they also get it right sometimes. When are they right and when are they wrong? 
\smallskip

Are they sometimes only partly right and partly wrong?  This course will explore science and physics through movies and television shows. Both the qualitative and quantitative aspects of science and physics will be explored however, students are only expected to use basic math and simple calculations (advanced math is \emph{not} required). 
\smallskip

Most of the course material will be presented in themes with each lasting about two weeks and involving a specific topic. There will be approximately 5 or 6 themes explored in this course. Possible theme topics include: Energy, momentum and conservation; gravity, forces and motion; the stars and the universe (cosmology); modern physics; size and scaling. Your teacher will discuss the themes to be covered during the first full week of classes. 
\smallskip

%\begin{itemize}
%\item The scientific method 
%\item How to describe motion
%\item Gravity, forces and projectile motion 
%\item Energy and laws of conservation 
%\item Atomic, nuclear and particle physics 
%\item Waves (sound waves, light waves, lasers, optics) 
%\item Properties of matter and phase transitions 
%\item Tunneling and quantum physics 
%\item The stars and the universe; cosmology
%\item Size and scaling 
%\item Time travel, causality and relativity 
%\item Means of transportation 
%\end{itemize}

While there is no laboratory component in this course, experimental verification of physical reality is a key aspect of science and as such some class activities will occasionally involve experimental demonstrations and the analysis of data collected in the class or acquired from film clips. No lab reports will be required for this course.
}
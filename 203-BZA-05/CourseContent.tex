\entry{Course content}{The material to be covered is contained in the following chapters and sections of the texts.
\begin{center}
  \begin{tabulary}{0.95\textwidth}{|c|L|L|}
    \hline
   Weeks & Topics & Content \\ \hline\hline
   	1--3 & Evolution of \mbox{Astronomical} Thought & Greek astronomy; the Copernican revolution; the contributions of Kepler and Galileo \\ \hline
	4--6 & Universal Gravitation & The Newtonian synthesis; orbital mechanics and the motion of planets, comets and spacecraft; 
      tides and precession \\ \hline
	5--6 & Earth, Moon, Sun and Sky & The seasons; time and the calendar; eclipses; celestial coordinate systems; navigation\\ \hline
	7--8 & Atoms and Starlight & The electromagnetic spectrum; blackbody radiation; spectral lines; the Doppler shift\\ \hline
	9--10 & Tools of the Astronomer & Visible-light telescopes and spectroscopes; radio, infrared, ultraviolet and X-ray astronomy\\ \hline
	11--12 & The Properties of Stars & The distances, motions, colours and brightnesses of the stars; stellar spectra, and what they can tell us; the Hertzsprung-Russell diagram; binary stars and stellar masses\\ \hline
	13--14 & The Evolution of Stars; Exotic Objects & How stars are born; the sources of energy in the stars; star clusters and their H-R diagrams; how stars die; red giants, white dwarfs, neutron stars and black holes
\\ \hline
	15 & Galaxies, Quasars and Cosmology & Our Milky Way Galaxy; a Universe of galaxies; the expanding Universe and the Hubble law; the age of the Universe; the primordial fireball; dark matter and dark energy; cosmological models; the ultimate fate of the Universe
\\ \hline
  \end{tabulary}
\end{center}
Some of the following labs will be performed: 
\begin{enumerate}
	\item Determining the orbit of Mars by Kepler's method
	\item The constellations -- finding your way around the sky
	\item Measuring the Moon's diameter at a lunar eclipse
	\item Finding the distance to the Crab Nebula
	\item Hubble's constant and the expansion of the Universe
	\item Classifying stellar spectra
\end{enumerate}
}
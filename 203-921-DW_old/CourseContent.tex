\entry{Course content}{The material to be covered is contained in the following chapters and sections of the text as well as the pdf files available to the students from the instructor.
\begin{center}
  \begin{tabulary}{0.95\textwidth}{|c|L|L|}
    \hline
    Weeks & Topics & Chapter \& Section \\ \hline\hline
   1--2 & Work and energy, kinetic energy, potential energy, energy transfer, power & Ch.x: x--x\\ \hline
	3--4 & Coulomb's law, electric field, electric potential & Ch.x: x--x\\ \hline
	5--6    & Capacitance, Ohm's law, resistivity, light bulbs & Ch.x: x--x\\ \hline
	7--8 & Kirchhoff's laws, series and parallel circuits, open and short circuits, power & Ch.x: x--x\\ \hline
	9--10 & RC circuits, the magnetic force, mass spectroscopy, inductance, solenoids & Ch.x: x--x \\ \hline
	11 & AC circuits and resonance& Ch.x: x--x \\ \hline
	12--13 & Electronic components: semiconductors, diodes, rectifiers, LED diodes, transistors & Ch.x: x--x \\ \hline
	14--15 & Transistor amplifier, oscillator, operational amplifier, solid state devices and logic circuits& Ch.x: x--x \\ \hline
  \end{tabulary}
\end{center}
The lab work is an integral part of the course. Ten labs will be performed and will be taken from the following topics: electric field, Ohm's law, series and parallel resistors, Kirchhoff's laws, RC circuits, resistivity, AC circuits, the diode rectifier, the transistor amplifier, oscillators, the operational amplifier, identification of components and schematic diagrams. Students might also be asked to work on group projects to be completed by the end of the semester.
}
